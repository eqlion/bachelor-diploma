\chapter{Модуляция сигнала в OWC}

Для цельной оптической беспроводной сети требуется повсеместное покрытие передающими элементами. Это означает использование больших количеств осветительных устройств с возможностью Li-Fi. Самые подходящие для этого варианты \--- некогерентные твердотельные LED из-за их низкой стоимости. Из-за физических свойств этих компонентов, информация может быть закодирована только в интенсивности излучаемого света, так как фаза и амплитуда света не могут быть модулированы. Это отличает VLC от РЧ коммуникаций.



Techniques such as on-off keying (OOK), pulse-position modulation
(PPM), pulse-width modulation (PWM) and unipolar M-ary pulse-amplitude modulation (M-PAM) can be
applied in a relatively straightforward fashion. As the modulation speeds are increased, however, these particular
modulation schemes begin to suffer from the undesired effects of intersymbol interference (ISI) due to the non-flat
frequency response of the OWC channel. Hence, a more resilient technique such as OFDM is required. OFDM
allows adaptive bit and energy loading of different frequency sub-bands according to the communication channel
properties.11 This leads to optimal utilization of the available resources. OFDM achieves the throughput capacity
in a non-flat communication channel even in the presence of nonlinear distortion.7