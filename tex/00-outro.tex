\Conclusion

Результатом выпускной квалификационной работы являются модель оптической системы в программе Zemax OpticStudio. В ходе работы был проведен обзор существующей коммерчески доступной компонентной базы, на примере производителя Thorlabs, выбран фотодетектор, построена модель оптической системы с ним, проведено исследование зависимости оптической мощности на нём в зависимости от различных параметров системы, был проведён эксперимент с целью рассчитать ширину полосу пропускания фотодиода. 

В ходе работы было показано, что возможно создание Li-Fi сети с использование стандартных компонентов, инфракрасной длины волны для восходящего сигнала, что является безопасным для человека, не мешает в освещении и позволяет увеличить мощность источника излучения для повышения качества передаваемого сигнала и скорости передачи информации.