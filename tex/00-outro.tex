\Conclusion

Результатом выпускной квалификационной работы являются модель оптической системы в программе Zemax OpticStudio. В ходе работы был проведен обзор существующей коммерчески доступной компонентной базы, на примере производителя Thorlabs, выбран фотодетектор, построена модель оптической системы с ним, проведено исследование зависимости оптической мощности на нём в зависимости от различных параметров системы, был проведён эксперимент с целью рассчитать ширину полосу пропускания фотодиода. 

В ходе работы было показано, что возможно создание Li-Fi сети с использованием стандартных компонентов, что позволит увеличить скорость внедрения и популяризации технологии, инфракрасной длины волны для восходящего сигнала, что является безопасным для человека, не мешает в освещении и позволяет увеличить мощность источника излучения для повышения качества передаваемого сигнала и скорости передачи информации. Была построена модель оптической системы, которая позволяет исследовать зависимость оптической мощности, улавливаемой фоточувствительной площадкой фотоприёмника в зависимости от различных параметров системы \--- углы расходимости источника, угол поворота источника, расстояние между источником и приёмником, мощность источника.

В дальнейшем работу можно продолжить: при использовании схемы согласования, возможно измерить ёмкость фотодиода и рассчитать его полосу пропускания, исследовать возможность и целесообразность использования светофильтров и просветляющего покрытия на оптике (фокусирующая линза, защитное стекло фотоприёмника) для улучшения качества связи или уменьшения необходимой мощности излучения источника.