\chapter{Экспериментальные исследования системы}
\label{ch:experiment}

Как уже было сказано в предыдущем разделе~\ref{ch:zemax}, в качестве приёмника был выбран фотодиод FDGA05. Рассчитаем пропускную способность $f_{BW}$ этого фотодиода по данным производителя~\cite{PDThorlabs}:

\begin{equation}
    f_{BW} = \frac{1}{2\pi R_L C_j},
\end{equation}

где $R_L = 50$ Ом \--- сопротивление нагрузки, $C_j = 10$ пФ $= 10^{-11}$ Ф \--- ёмкость фотодиода. Тогда $f_{BW} = 3.1831 \cdot 10^8~\text{Гц} = 318.31~\text{МГц}$.

\begin{figure}[!h]
    \centering
    \includegraphics[width=.55\textwidth]{inc/img/experimental_scheme.png}
    \caption{Схема экспериментальной установки, использованной для проведения измерения ёмкости фотодиода FDGA05. Здесь ЛД и ФД \--- лазерный и фото диоды соответственно}
    \label{fig:experimental_scheme}
\end{figure}

\begin{figure}[!h]
    \centering
    \includegraphics[width=.2\textwidth]{inc/img/pd.jpg}
    \caption{Фотодиод FDGA05, распаянный на плате с SMA разъёмом, вид спереди}
    \label{fig:pd_sma}
\end{figure}

\begin{figure}[!h]
    \centering
    \includegraphics[width=.55\textwidth]{inc/img/ld_and_pd.jpg}
    \caption{Фотодиод FDGA05 (слева) и лазерный диод FPL1055T (справа), распаянные на платах с SMA разъёмом, вид сверху}
    \label{fig:ld_pd_sma}
\end{figure}

Экспериментально подтвердим это значение. Для этого соберём экспериментальную установку, состоящую из фотодиода FDGA05~\cite{PDThorlabs} и лазерного диода FPL1055T~\cite{LDThorlabs}. Оба компоненты распаяны на платы (рисунки~\ref{fig:pd_sma},~\ref{fig:ld_pd_sma}) для подключения к векторному анализатору цепей Rohde & Schwarz ZVA 40 через SMA разъем (собранная установка аналогична установке представлена на рисунке~\ref{fig:experiment_setup_photo}: на ней показаны лазерный диод FPL1055T слева и фотодиод, использованный в работе~\cite{Kozyreva2019}, схематично экспериментально показана на рисунке~\ref{fig:experimental_scheme}).

\begin{figure}[!h]
    \centering
    \includegraphics[width=.9\textwidth]{inc/img/experiment_setup.jpg}
    \caption{Собранная экспериментальная установка с лазерным диодом FPL1055T и фотодиодом, использованным в работе~\cite{Kozyreva2019}}
    \label{fig:experiment_setup_photo}
\end{figure}

К первому порту векторного анализатора цепей подключен лазерный диод, ко второму \--- фотодиод. Для расчёта пропускной способности необходимо измерить ёмкость фотодиода экспериментально. Сделаем это при помощи измерения комплексного коэффициента отражения векторным анализатором (полученная диаграмма Смита приведена на рисунке~\ref{fig:smith_plot}).

\begin{figure}[!h]
    \centering
    \includegraphics[width=.9\textwidth]{inc/img/smith_plot.jpg}
    \caption{Диаграмма Смита для фотодиода}
    \label{fig:smith_plot}
\end{figure}

По измеренным данным не удаётся демодулировать сигнал из-за большого коэффициента отражения. Это связано с тем, что в сверх высоко частотный тракт ничего не проходит, так как нет согласования фотодиода по высоким частотам. С точки зрения работы в высокочастотном тракте, фотодиод представляет собой рассогласованную по сопротивлению нагрузку, поэтому сигнал переотражается и без схемы согласования не удаётся детектировать высокочастотный сигнал и исследовать амплитудно-частотную характеристику.