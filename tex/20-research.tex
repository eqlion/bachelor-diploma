\chapter{Генерация и применение ЧГ}
\label{ch:research}

\Abbrev{FP}{Fabry-P\'ero \--- Фабри-Перо}
\Abbrev{DFB}{Distributed feedback \--- распределенная обратная связь}
Самым частым способом получения ЧГ является использование лазеров в режиме синхронизации мод, так как для создания ЧГ необходимы фемтосекундные импульсы. Другой метод основан на одном или на каскаде нескольких внешних модуляторов, что требует относительно сложной настройки и в результате получаются высокие потери. Недавно был продемонстрирован простой и дешёвый источник ЧГ, основанный на переключении усиления лазерного диода с внешней инжекцией в Фабри-Перо. Тем не менее, переключение усиления в лазерном диоде с распределённой обратной связью является наиболее простым и надёжным способом генерации импульсов, так что имеет смысл исследовать генерацию ЧГ в DFB лазере~\cite{HuataoZhu2016}. 

В \cite{HuataoZhu2016} было показана экспериментальная установка для генерации ЧГ (рисунок \ref{fig:HuataoZhu}). Он состоит из настраиваемого лазера (TLS), источника радиочастоты и DFB лазера. TLS является главным лазером, и он подсоединен к контроллеру поляризации (PC) для настройки состояния поляризации и к регулируемому оптическому аттенуатору (VOA) для регулировки мощности. Источник радиочастот генерировал синусный сигнал с частотой 10 ГГц, который проходил через усилитель с усилением в 23 дБ, таким образом напрямую модулируя DFB лазер. В конце подключен оптический спектральный анализатор (OSA) для мониторинга оптического спектра.

\begin{figure}[ht]
  \centering
  \includegraphics[width=.8\textwidth]{inc/img/zhu.png}
  \caption{Схематичная диаграмма генератора ЧГ, основанного на инжекционном DFB лазере~\cite{HuataoZhu2016}. Здесь OFCG \--- генератор ЧГ, SG \--- генератор радиочастотного сигнала, Amp \--- усилитель радиочастотного сигнала, DFB \--- распределённая обратная связь, TLS \--- настраиваемый лазерный источник, PC \--- контроллер поляризации, VOA \--- регулируемый оптический аттенуатор, OSA \--- оптический спектральный анализатор.}
  \label{fig:HuataoZhu}
\end{figure}

В результате проведённого эксперимента было установлено, что предложенная конструкция генератора ЧГ может создавать 10 спектральных линий с плоскостью 1.5 дБ. При добавлении в схему фазового модулятора удалось достичь 14 плоских спектральных линий. Предлагаемая схема может быть использована в когерентной оптической коммуникации и микроволновой фотонике.

В \cite{Quirce2018} была исследована поляризационная динамика VCSEL с центральной длиной волны $\lambda=1550$ нм с модуляцией переключения амплитуды усиления для получения ЧГ. Было показано, что благодаря генерации двух перпендикулярных линейно поляризованных гребёнок, результирующая общая ЧГ гораздо шире, чем та, которая была получена из основной поляризации. 

Авторы статьи \cite{Quirce2020} исследовали нелинейную динамику ЧГ, сгенерированных полупроводниковым лазером с оптическими инжекциями. В их эксперименте генератор ЧГ состоял из основного лазера, излучение которого инжектировалось в дополнительный лазер с помощью оптического циркулятора. Количество инжектируемой мощности контролировалось с помощью регулируемого оптического аттенуатора, подключенного к выходу основного лазера, как и в случае выше. Общая схема эксперимента аналогична схеме, представленной на рисунке \ref{fig:HuataoZhu}. Качество полученной ЧГ характеризуется как десяти децибельная спектральная ширина \--- максимальное частотное разделение между пиками со значениями больше 10 дБ ниже, чем абсолютный максимум оптического спектра. 

\begin{figure}[ht]
  \centering
  \includegraphics[width=\textwidth]{inc/img/quirce.png}
  \caption{Оптический спектр полупроводникового лазера с переключением усиления, при токе смещения $I_{bias} = 30$ мА, частота модуляции $f_m = 5$ ГГц, $V_{RF}=0.5$ В, (a) $\Delta v = 6.5$ ГГц и $P_{inj}=417.3$ мкВт, (b) $\Delta v = -0.5$ ГГц и $P_{inj}=330.1$ мкВт, (c) $\Delta v = -1$ ГГц и $P_{inj}=210.5$ мкВт, (d) $\Delta v = -1$ ГГц и $P_{inj}=329.5$ мкВт, (e) $\Delta v = 6.5$ ГГц и $P_{inj}=332.1$ мкВт. Стрелка показывает частоту оптической инжекции~\cite{Quirce2020}.}
  \label{fig:quirce}
\end{figure}

Вообще, для высокоточных лазерных измерений частоты, при которых напрямую измеряется абсолютная оптическая частота, требуются большие газовые лазеры и точечные контактные диоды с коротким периодом жизни, такие как металл-изолятор-металл\--диоды. Кроме того, для системы когерентной оптической коммуникации необходимо измерять разницу частот лазера, которые находятся в диапазоне до нескольких терагерц. Несмотря на это, измерение такой большой разницы частот может вызывать сложности. Относительно недавно был предложен \cite{Kourogi1993} компактный высокоточный измеритель частот на длине волны 1.5 мкм, работа которого основана на генераторе ЧГ. Схема предлагаемого измерителя частот представлена на рисунке~\ref{fig:kourogi}: система состоит из двух частей \--- высокоточного референтного лазерного диода ($LD_R$) на длине волны 1.55 мкм и генератора ЧГ для измерения разницы частот между референтным лазером и измеряемым лазером ($LD_x$), измеряемая разница частот может достигать порядка ТГц.

\begin{figure}[ht]
  \centering
  \includegraphics[width=\textwidth]{inc/img/kourogi.png}
  \caption{Предлагаемая в~\cite{Kourogi1993} система измерения оптической частоты для 1.5 мкм полупроводникового лазера.}
  \label{fig:kourogi}
\end{figure}

Частота референтного лазера фиксирована $\nu_R=(\nu_1 - \nu_2)/2$ при помощи генерации световой волны как суммы и разницы частот референтного лазера и твердотельного лазера на длине волны 1.06 мкм ($\nu_S$), и с помощью фиксирования суммы и разницы частот на частотах ($\nu_1$ и $\nu_2$) двух стабилизированных He-Ne лазеров, где $\nu_1$ и $\nu_2$ \--- абсолютные частоты этих лазеров. Можно использовать LiNbO3 для сложения и вычитания частот. Так как гетеродинное измерение очень чувствительное, не требуется высокая мощность сигналов.

Генератор ЧГ генерирует модуляционные боковые полосы из падающего лазерного излучения, и эти боковые полосы используются как локальные осцилляторы для измерения разницы частот между референтным лазером и измеряемым лазером. Точность измерения разницы частот может быть равна частоте модуляции. Для высокой точности измерения необходим ``широкий'' генератор ЧГ. На рисунке~\ref{fig:kourogi2} представлена его конструкция. Он состоит из электро-оптического фазового модулятора, установленного в резонатор Фабри-Перо. Модулятор является покрытым антибликовым покрытием кристаллом LiNbO3, установленным в микроволновый волновод. Для того, чтобы он был высокоэффективным для высоких частот, ширина волновода была выбрана для резонанса микроволн, чтобы концентрировать микроволны в кристалле. 

\begin{figure}[ht]
  \centering
  \includegraphics[width=\textwidth]{inc/img/kourogi2.png}
  \caption{Конструкция генератора ЧГ, использованного в в~\cite{Kourogi1993}.}
  \label{fig:kourogi2}
\end{figure}

\Abbrev{WDM}{Wavelength division multiplexed \--- мультиплексирование с разделением по длине волны}
\Abbrev{OTDM}{Optical time division multiplexed \--- мультиплекирование с разделением по оптическму времени}
\Abbrev{MLL}{Modelocked laser \--- лазер в режиме синхронизации мод}
В статье \cite{Delfyett2006} авторы показывают потенциал применения ЧГ в коммуникациях с ультравысокой пропускной способностью и процессинге сигналов как показано на рисунке \ref{fig:delfyett}. Этот рисунок показывает как оптические ЧГ могут быть использованы в:

\begin{enumerate}
  \item Аналоговом и цифровом WDM формате, когда каждая из полос гребёнки модулируется независимо и когерентно детектируется при помощи гребёнок отдельного синхронизированного приёмного MLL;
  \item Ультравысоко скоростные форматы OTDM, при которых высокоскоростные последовательности импульсов демультиплексируются по времени с помощью синхронизированного приёмника MLL;
  \item Мультиплексирование с разделением по оптическому коду, при котором переданные импульсы модулируются при помощи включения\--выключения и закодированы спектральной фазой, а приём осуществляется когерентным гомодинным детектором при помощи синхронизированного приёмника MLL.
\end{enumerate}

\begin{figure}[ht]
  \centering
  \includegraphics[width=\textwidth]{inc/img/delfyett.png}
  \caption{Использование ЧГ для модуляции в временной области, частотной области и кодовой области~\cite{Delfyett2006}.}
  \label{fig:delfyett}
\end{figure}

В итоге авторы показали, что частотные гребёнки полученные при помощи MLL являются идеальными оптическими синусоидами для большого количества различных применений в когерентной обработке сигналов, особенно для генерации произвольных длин волн и для архитектуры когерентного приёмника. Ключевой потребностью при таком применении является стабильность и надёжность генератора оптических ЧГ. 

\Abbrev{MZ}{Mach-Zender \--- [модулятор/интерферометр] Маха-Цендера}
В другой статье \cite{Yokota2015} авторы исследуют ЧГ, сгенерированную InP полупроводниковым MZ-модулятором с помощью численной модели, которая принимает во внимание нелинейность изменения коэффициента преломления и оптическое поглощение, вызванное напряжением, приложенным к модулятору. Эта нелинейность крайне мала для обычных MZ-модуляторов на основе LiNbO3, но становятся заметными для полупроводниковых MZ-модуляторов. 

Роутинг на основе WDM является незаменимой технологией для сетевых фотонных систем следующего поколения, в которых длина волны определяет пункт назначения сигнала. Такая система требует многоканальный оптический передатчик, который можно получить при комбинации WDM-модулятора и генератора ЧГ. В этой работе исследователи описывают девяти-канальную ЧГ, сгенерированную InP-MZ модулятором с помощью численных методов. Схематичное изображение экспериментальной установки и ЧГ приведено на рисунке \ref{fig:yokota}

\begin{figure}[ht]
  \centering
  \includegraphics[width=\textwidth]{inc/img/yokota.png}
  \caption{(a) Схема экспериментальной установки и (b) ЧГ~\cite{Yokota2015}.}
  \label{fig:yokota}
\end{figure}

В другой статье \cite{Yokota_2015} эти же авторы численную модель модулятора Маха-Цендера, который показан на рисунке \ref{fig:yokota}: постоянное напряжение $V_{DC1}$ и $V_{DC2}$ был приложен к плечам №1 и №2 соответственно. Помимо него к обоим плечам было приложено переменное напряжение $V_{AC}$ в виде синусоидального радиочастотного напряжения. Напряжение на первом плече было меньше, чем на втором. Тогда напряжения $V_1$ и $V_2$ можно записать следующим образом:

\begin{equation}
  V_1 = aV_{AC}\sin(\omega_m t) + V_{DC1}
\end{equation}

\begin{equation}
  V_2 = -V_{AC}\sin(\omega_m t) + V_{DC2}
\end{equation}

где $\omega_m$ \--- угловая частота переменного тока. Когда $V_1$ и $V_2$ приложены модулятору, выходное электрическое поля модулятора $E_o$ записывается следующим образом:

\begin{equation}
  E_o = \frac{E_i}{2} \left[ c_1(V_1)e^{j\{\omega t - \phi_1(V_1) - \Delta \phi\}} + c_2(V_2) e^{j\{\omega t-\pi_2(V_2)\}} \right]
\end{equation}

где $\phi_1$ и $\phi_2$ \--- изменения фазы, а $c_1$ и $c_2$ \--- коэффициенты, определяемые оптическим поглощением в каждом плече. 

\begin{figure}[ht]
  \centering
  \includegraphics[width=\textwidth]{inc/img/yokota2.png}
  \caption{(a) Девиация интенсивности девятиканальной ЧГ и (b) соответствующее напряжение по отношению к переменному напряжению. Кругами, треугольниками и квадратами обозначены измеренные значения. Сплошными, пунктирными и точечными линиями обозначены вычисленные значения~\cite{Yokota_2015}.}
  \label{fig:yokota2}
\end{figure}

При сравнении измеренных значений с вычисленными по модели (см. рис. \ref{fig:yokota2}) хорошо видно, что модель хорошо описывает экспериментальные измерения. Важно отметить, что это значения не совпадают только в случае, когда принимается во внимание линейная часть изменения показателя преломления в модуляторе. Эта модель показывает, что нельзя пренебрегать нелинейностью показателя преломления и оптическим поглощением в полупроводниках при оценки плоскости ЧГ.

Важно также отметить и применение ЧГ в спектроскопии, так как с помощью них можно значительно улучшить существующие интерферометры Майкельсона с Фурье преобразованием. Создаётся новый класс инструментов \--- спектрометры с двойной гребёнкой, в них нет движущихся частей, что позволяет получить более быстрые и точные измерения для больших спектральных интервалов~\cite{Picqu2019}.

В большинстве случаев, генератор ЧГ является широкополосным источником света, который одновременно возбуждает несколько (много) переходов в исследуемом образце. Тогда необходим спектрометр (за исключением случаев с маленькой шириной ЧГ или с очень простым спектром). Если спектрометр имеет достаточное разрешение, можно рассматривать отдельные линии ЧГ, что позволяется использовать само-калибровку по частоте. Тогда разрешение определяется как частота повторения гребёнки $f_{rep}$, несмотря на то, что спектральные элементы могут быть определены со значительно более высокой точностью. Когда разрешение спектрометра равно (или лучше) расстоянию между линиями в гребёнке, инструментальная форма линии начинает определяться шириной линий гребёнки, а не откликом спектрометра. Когда ширина линий гребёнки становится \'уже, вклад инструментальной формы линии становится пренебрежимо малым, так как атомные или молекулярные резонансы имеют ширину равную (или большую), чем расстояние между линиями в гребёнке $f_{rep}$. 

Виды спектроскопических техник с использованием ЧГ (все техники представлены на рисунке~\ref{fig:picqu}):

\begin{itemize}
  \item прямая ЧГ спектроскопия \--- (рис.~\ref{fig:picqu}a) является самым простым подходом к линейной или нелинейной ЧГ спектроскопии. Для линейной спектроскопии одна линия гребёнки резонирует с переходом, а все остальные линии, в идеале, не должны совпадать с резонансами. Для двух-фотонной спектроскопии, много пар линий ЧГ с одинаковой суммой частот делают вклад в возбуждение, что является настолько же эффективным, как и возбуждение непрерывным лазерным пучком с такой же средней мощностью; 
  \item спектроскопия Рамзи гребёнок \--- (рис.~\ref{fig:picqu}b) измеряет интерференцию между возбуждениями атомных или молекулярных образцов с помощью двух интенсивных задержанных во времени импульсов, выведенных из ЧГ;
  \item спектроскопия с использованием дисперсионного спектрометра \--- (рис.~\ref{fig:picqu}c) является простым и надёжным инструментом для мультиканальной параллельной широкополосной спектроскопии с ЧГ. Было успешно продемонстрировано использование решёток со сканирующими детекторами или камерами;
  \item основанная на интерферометре Майкельсона Фурье спектроскопия \--- (рис.~\ref{fig:picqu}d) является наиболее успешной техникой спектроскопии на протяжении уже более 50 лет. Обычно используется некогерентный источник света, а спектрометр измеряет интерференцию двух сигналов из двух плеч с разным оптическим ходом. При использовании ЧГ в качестве источника света, частота каждой линии сдвигается при отражении в движущимся плече интерферометра на $-2(nf_{rep}+f_0)v/c$. Когерентный источник излучения, такой как генератор ЧГ, имеет более высокую яркость, что приводит к возрастанию отношения сигнала к шуму и снижению времени измерения;
  \item двух гребёночная спектроскопия \--- (рис.~\ref{fig:picqu}e) новый подход к Фурье спектроскопии без движущихся частей. В большинстве реализаций ЧГ с частотой повторения $f_{rep}$ после прохождения через образец сталкивается со второй гребёнкой с чуть другой частотой повторения $f_{rep}+\delta f_{rep}$, которая работает как локальный осциллятор. Интерференционный сигнал записывается как функция от времени, после чего производится Фурье преобразование для получения спектра. Фундаментальное отличие между двух гребёночными спектрометрами и традиционными заключается в свободе геометрии. С рассеивающим или интерференционным инструментом теоретическая разрешающая способность может быть записана как $R = \Delta/\lambda$, где $\Delta$ \--- максимальная разность оптического пути, а $\lambda$ \--- длина волны. С двух гребёночным интерферометром разрешающая способность записывается иначе: $R = T/\tau$, где $T$ \--- время измерения, а $\tau$ \--- период световой вибрации. Таким образом, двух гребёночная спектроскопия является единственной техникой, которая потенциально может достичь разрешения равного интервалу между линиями гребёнки, что позволяет открыть новые возможности для прецизионной спектроскопии и метрологии. 
\end{itemize}

\begin{figure}[ht]
  \centering
  \includegraphics[width=.6\textwidth]{inc/img/picqu.png}
  \caption{Методы спектроскопии с использованием ЧГ: (а) прямая ЧГ спектроскопия, (b) спектроскопия Рамзи гребёнок, (с) спектроскопия с использованием дисперсионного спектрометра, (d) основанная на интерферометре Майкельсона Фурье спектроскопия, (e) двух гребёночная спектроскопия~\cite{Picqu2019}.}
  \label{fig:picqu}
\end{figure}