\sloppy

% Настройки стиля ГОСТ 7-32
% Для начала определяем, хотим мы или нет, чтобы рисунки и таблицы нумеровались в пределах раздела, или нам нужна сквозная нумерация.
\EqInChapter % формулы будут нумероваться в пределах раздела
\TableInChapter % таблицы будут нумероваться в пределах раздела
\PicInChapter % рисунки будут нумероваться в пределах раздела

% Добавляем гипертекстовое оглавление в PDF
\usepackage[
bookmarks=true, colorlinks=true, unicode=true,
urlcolor=black,linkcolor=black, anchorcolor=black,
citecolor=black, menucolor=black, filecolor=black,
]{hyperref}

\AfterHyperrefFix

\usepackage{microtype}% полезный пакет для микротипографии, увы под xelatex мало чего умеет, но под pdflatex хорошо улучшает читаемость

% Тире могут быть невидимы в Adobe Reader
\ifInvisibleDashes
\MakeDashesBold
\fi

\usepackage{graphicx}   % Пакет для включения рисунков

% С такими оно полями оно работает по-умолчанию:
% \RequirePackage[right=10mm,top=20mm,left=20mm,bottom=20mm,headsep=0pt,includefoot]{geometry}
% Если вас тошнит от поля в 10мм --- увеличивайте до 20-ти, ну и про переплёт не забывайте
\geometry{ignorefoot} % считать от нижней границы текста


% Пакет Tikz
\usepackage{tikz}
\usetikzlibrary{arrows,positioning,shadows}

% Произвольная нумерация списков.
\usepackage{enumerate}

% ячейки в несколько строчек
\usepackage{multirow}

% Для рисования иерархии папок - forest
\usepackage{forest}

\usepackage{fontspec} 


\defaultfontfeatures{Ligatures={TeX},Renderer=Basic}

\setmainfont{Times New Roman}
\newfontfamily\cyrillicfont{Times New Roman}
% \setsansfont{Comic Sans MS}
\setmonofont{Courier New}
\newfontfamily\cyrillicfonttt{Courier New}

% Стиль по умолчанию
\forestset{
  default preamble={
    for tree={
      font=\ttfamily,
      grow'=0,
      child anchor=west,
      parent anchor=south,
      anchor=west,
      calign=first,
      edge path={
        \noexpand\path [draw, \forestoption{edge}]
        (!u.south west) +(7.5pt,0) |- node[fill,inner sep=0pt] {} (.child anchor)\forestoption{edge label};
      },
      before typesetting nodes={
        if n=1
        {insert before={[,phantom]}}
        {}
      },
      fit=band,
      before computing xy={l=15pt},
    }
  }
}

% itemize внутри tabular и выравнивание таблиц
\usepackage{paralist,array,tabularx}

% Немного снимаем боль при работе с таблицами, которые надо центрировать
\newcolumntype{L}{>{\raggedright\arraybackslash}l}
\newcolumntype{P}[1]{>{\raggedright\arraybackslash}p{#1}}
\newcolumntype{M}[1]{>{\centering\arraybackslash}m{#1}}

% tabularx
\newcolumntype{Z}{>{\centering\arraybackslash}X}

%\setlength{\parskip}{1ex plus0.5ex minus0.5ex} % разрыв между абзацами
\setlength{\parskip}{0ex} % разрыв между абзацами
\usepackage{blindtext}

% Центрирование подписей к плавающим окружениям
%\usepackage[justification=centering]{caption}

% Отступ после листинга
\captionsetup[listing]{belowskip=20pt}

% Название таблицы перед таблицей
\usepackage{floatrow}
\floatsetup[table]{capposition=top}