\begin{center}
    \hfill \break
    \footnotesize{\textbf{Министерство образования и науки Российской Федерации}} \\
    \scriptsize{\textbf{ФЕДЕРАЛЬНОЕ ГОСУДАРСТВЕННОЕ АВТОНОМНОЕ ОБРАЗОВАТЕЛЬНОЕ УЧРЕЖДЕНИЕ ВЫСШЕГО ОБРАЗОВАНИЯ}} \\
    \normalsize{\textbf{«Университет ИТМО»}} \\
\end{center}

\noindent\textbf{Факультет} лазерной фотоники и оптоэлектроники

\noindent\textbf{Образовательная программа} Лазеры для информационно-коммуникационных систем

\noindent\textbf{Направление подготовки (специальность)} 12.03.05 Лазерная техника и лазерные технологии\hfill\break

\begin{center}
    \Large{О Т Ч Е Т} \\
    \normalsize{о производственной преддипломной практике}
\end{center}


\hfill\break

\noindent
Тема задания: Разработка фотоприемной части восходящего канала связи по технологии Li-Fi \\

\noindent
Обучающийся: Лапа Николай Андреевич L3430 \\

\noindent
Руководитель практики от университета: Козырева Ольга Андреевна, ассистент института перспективных систем передачи данных Университета ИТМО \\

\hfill \break
\hfill \break
\hfill\break
\hfill\break
\hfill\break
\begin{center}
    Санкт-Петербург \\
    2021
\end{center}
\thispagestyle{empty}