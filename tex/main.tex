%% Преамбула TeX-файла

% 1. Стиль и язык
\documentclass[liberation]{G7-32}

% Остальные стандартные настройки убраны в preamble.inc.tex.
\sloppy

% Настройки стиля ГОСТ 7-32
% Для начала определяем, хотим мы или нет, чтобы рисунки и таблицы нумеровались в пределах раздела, или нам нужна сквозная нумерация.
\EqInChapter % формулы будут нумероваться в пределах раздела
\TableInChapter % таблицы будут нумероваться в пределах раздела
\PicInChapter % рисунки будут нумероваться в пределах раздела

% Добавляем гипертекстовое оглавление в PDF
\usepackage[
bookmarks=true, colorlinks=true, unicode=true,
urlcolor=black,linkcolor=black, anchorcolor=black,
citecolor=black, menucolor=black, filecolor=black,
]{hyperref}

\AfterHyperrefFix

\usepackage{microtype}% полезный пакет для микротипографии, увы под xelatex мало чего умеет, но под pdflatex хорошо улучшает читаемость

% Тире могут быть невидимы в Adobe Reader
\ifInvisibleDashes
\MakeDashesBold
\fi

\usepackage{graphicx}   % Пакет для включения рисунков

% С такими оно полями оно работает по-умолчанию:
% \RequirePackage[right=10mm,top=20mm,left=20mm,bottom=20mm,headsep=0pt,includefoot]{geometry}
% Если вас тошнит от поля в 10мм --- увеличивайте до 20-ти, ну и про переплёт не забывайте
\geometry{ignorefoot} % считать от нижней границы текста


% Пакет Tikz
\usepackage{tikz}
\usetikzlibrary{arrows,positioning,shadows}

% Произвольная нумерация списков.
\usepackage{enumerate}

% ячейки в несколько строчек
\usepackage{multirow}

% Для рисования иерархии папок - forest
\usepackage{forest}

\usepackage{fontspec} 


\defaultfontfeatures{Ligatures={TeX},Renderer=Basic}

\setmainfont{Times New Roman}
\newfontfamily\cyrillicfont{Times New Roman}
% \setsansfont{Comic Sans MS}
\setmonofont{Courier New}
\newfontfamily\cyrillicfonttt{Courier New}

% Стиль по умолчанию
\forestset{
  default preamble={
    for tree={
      font=\ttfamily,
      grow'=0,
      child anchor=west,
      parent anchor=south,
      anchor=west,
      calign=first,
      edge path={
        \noexpand\path [draw, \forestoption{edge}]
        (!u.south west) +(7.5pt,0) |- node[fill,inner sep=0pt] {} (.child anchor)\forestoption{edge label};
      },
      before typesetting nodes={
        if n=1
        {insert before={[,phantom]}}
        {}
      },
      fit=band,
      before computing xy={l=15pt},
    }
  }
}

% itemize внутри tabular и выравнивание таблиц
\usepackage{paralist,array,tabularx}

% Немного снимаем боль при работе с таблицами, которые надо центрировать
\newcolumntype{L}{>{\raggedright\arraybackslash}l}
\newcolumntype{P}[1]{>{\raggedright\arraybackslash}p{#1}}
\newcolumntype{M}[1]{>{\centering\arraybackslash}m{#1}}

% tabularx
\newcolumntype{Z}{>{\centering\arraybackslash}X}

%\setlength{\parskip}{1ex plus0.5ex minus0.5ex} % разрыв между абзацами
\setlength{\parskip}{0ex} % разрыв между абзацами
\usepackage{blindtext}

% Центрирование подписей к плавающим окружениям
%\usepackage[justification=centering]{caption}

% Отступ после листинга
\captionsetup[listing]{belowskip=20pt}

% Название таблицы перед таблицей
\usepackage{floatrow}
\floatsetup[table]{capposition=top}

% Настройки листингов.
\usepackage{local-minted}

% Полезные макросы листингов.
% Любимые команды
\newcommand{\kt}[1]{\mintinline{kotlin}|#1|}
\newcommand{\codeline}[2]{\mint[style=bw]{#1}|#2|}
\newcommand{\code}[1]{\texttt{#1}}
\newcommand{\todo}[1]{{\color{red}TODO: {#1}}}
\newcommand{\conclusions}{
  \backmatter
  \section{Выводы по разделу}
  \mainmatter
}


% Стиль титульного листа и заголовки


\begin{document}
  \setcounter{page}{7}

  % \begin{center}
    \hfill \break
    \footnotesize{\textbf{Министерство науки и высшего образования Российской Федерации}} \\
    \scriptsize{\textbf{ФЕДЕРАЛЬНОЕ ГОСУДАРСТВЕННОЕ АВТОНОМНОЕ ОБРАЗОВАТЕЛЬНОЕ УЧРЕЖДЕНИЕ ВЫСШЕГО ОБРАЗОВАНИЯ}} \\
    \normalsize{\textbf{«НАЦИОНАЛЬНЫЙ ИССЛЕДОВАТЕЛЬСКИЙ \\ УНИВЕРСИТЕТ ИТМО»}} \\
    \hfill \break
    \normalsize{\textbf{ФАКУЛЬТЕТ ЛАЗЕРНОЙ ФОТОНИКИ И ОПТОЭЛЕКТРОНИКИ}} \\
    \hfill \break
    \normalsize{Дисциплина} \\
    \normalsize{<<Полупроводниковые лазеры>>} \\
\end{center}
\hfill \break
\begin{center}
    \large{Реферат} \\
    \normalsize{Генерация оптических частотных гребенок в полупроводниковых лазерах}\\
\end{center}
\hfill \break
\hfill \break
\hfill \break
\begin{flushright}
    Выполнил: \\
    студент группы L3430 \\
    Лапа Н. \\
    Преподаватель: \\
    Ковалёв А. В.
\end{flushright}
\hfill \break
\hfill \break
\hfill \break
\hfill \break
\hfill \break
\begin{center} Санкт-Петербург \\ 2020 \end{center}
\thispagestyle{empty}
\newpage
  % Предварительные приготовления
%   
% Источники, которые будут помещены в верх списка
\nocite{methodic}

% Переносы слов
\hyphenation{МИЭТе}


  \frontmatter % выключает нумерацию ВСЕГО; здесь начинаются ненумерованные главы: реферат, введение, глоссарий, сокращения и прочее.

  % \maketitle %создает титульную страницу

  %\listoffigures                         % Список рисунков

  %\listoftables                          % Список таблиц

  %\NormRefs % Нормативные ссылки
  % Команды \breakingbeforechapters и \nonbreakingbeforechapters
  % управляют разрывом страницы перед главами.
  % По-умолчанию страница разрывается.

  % \nobreakingbeforechapters
  % \breakingbeforechapters

  \tableofcontents

  \printnomenclature % Автоматический список сокращений

  \Introduction

\Abbrev{РЧ}{радио частоты}
\Abbrev{VLC (visible light communication)}{связь по видимому свету}

Спустя тридцать лет после появления первых коммерческих мобильных коммуникационных систем, беспроводная связь эволюционировала в обыкновенное удобство как газ или электричество. Экспоненциальный рост в мобильном трафике в течение последних трёх десятилетий привет к масштабному разворачиванию беспроводных систем. Как следствие, ограниченный доступный РЧ-диапазон является объектом постоянного переиспользования и межканальной интерференции, что значительно ограничивает ёмкость сетей. Таким образом, было много различных предупреждений о грядущем <<кризисе РЧ спектра>>~\cite{Ofcom2013}, так как требования к передачи мобильных данных данных продолжают расти, в то время как спектральная эффективность сетей насыщается, даже несмотря на введение новых стандартов и значительный технологический прогресс в этой области. По оценкам, к 2022 году более $77$ эксабайтов ($10^{60}$ байт) трафика будет передаваться через мобильные устройства каждый месяц (около одного зеттабайта в год)~\cite{Cisco2019}. В середине прошлого десятилетия было предложено использование связи по видимому свету (VLC) в качестве потенциального решения для избежания <<кризиса РЧ спектра>>.

\Abbrev{ИК}{инфракрасный}
\Abbrev{LED (Light emitting diode)}{светодиод}
\Abbrev{AP (Access point)}{точка доступа}
\Abbrev{PLC (Power-line communication)}{связь по электросети}
\Abbrev{PoE (Power over Ethernet)}{питание через Ethernet}


В течение прошлого десятилетия значительные усилия были направлены на изучение альтернативных частей электро-магнитного спектра, которые потенциально смогут разгрузить б\'ольшую часть трафика из загруженного РЧ диапазона. Было продемонстрировано использование миллиметровых волн в коммуникации в 28 ГГц  диапазоне, а так же использование видимого и ИК света. Это особенно полезно, так как освещение \--- удобство, которое имеется практически в любой жилой среде и для которого существует готовая инфраструктура. Использование видимого света для высокоскоростных линий связи становится возможно из-за использования LED. В этом смысле концепт комбинирования функций освещения и коммуникации позволяет экономить энергию (и деньги) и сократить углеродный след. Во-первых, установка точек доступа (AP) является достаточно тривиальной задачей, так как можно переиспользовать уже существующую инфраструктуру с использованием готовых технологий, таких как связь по электросети (PLC) и питание через Ethernet (PoE). Во-вторых, так как освещение обычно работает в помещениях даже в течение светлого времени суток, дополнительное питание передатчиков будет незначительным. Помимо этого, видимый свет включает в себя сотни ТГц свободного канала, что на четыре порядка больше, чем полный РЧ спектр до 30 ГГц, включая миллиметровый спектр. Оптическое излучение, в общем, не интерферирует с другими радио волнами и не мешает работе чувствительного электрического оборудования. Таким образом, свет идеален для беспроводного покрытия в местах, чувствительных к электро-магнитному излучению (например, больницы, самолёты, топливно-химические и атомные электростанции и другие). Помимо этого, так как свет не может проникать через непрозрачные поверхности (стены), создается более высокий уровень безопасности соединения. Эта же особенность может быть использована для избежания интерференции между двумя соседними сетями.

\Abbrev{RGB (Red-green-blue)}{красный-синий-зелёный}
\Abbrev{MIMO (Multiple-input-mulitple-outputs)}{несколько входов и выходов}
\Abbrev{IM/DD (Intensity modulation and direct detection)}{модуляция интенсивности и прямое детектирования}
\Abbrev{OFDM (Orthogonal frequency division multiplexing)}{мультиплексирование
с ортогональным частотным разделением каналов}
\Abbrev{OWC (Optical wireless communication)}{оптическая беспроводная коммуникация}
\Abbrev{Li-Fi}{light fidelity}
\Abbrev{Wi-Fi}{wireless fidelity}

В течение последних десяти лет, было проведено значительное количество исследований об улучшении скорости передачи между двумя устройствами с использованием существующих светодиодов в лабораторных условиях. В 2012 году была достигнута скорость передачи данных выше 1 Гб/с с использованием белых фосфорных LED~\cite{Khalid2012}, и 3.4 Гб/с с помощью красно-сине-зелёного (RGB) LED~\cite{Cossu2012}. Также была продемонстрирована~\cite{Azhar2013} схожая гигабитная система с белым фосфорным LED в виде матрицы 4 на 4 в конфигурации несколько входов и выходов (MIMO). Теоретическая структура для достижимой ёмкости модуляции интенсивности и прямого детектирования (IM/DD) с использованием мультиплексирования с ортогональным частотным разделением каналов (OFDM) была показана в~\cite{Dimitrov2013}. Для успешной реализации системы мобильной связи необходима готовая сетевая система. Это и есть то, что называется Li-Fi \--- сетевое мобильное высокоскоростное VLC решение для беспроводной связи~\cite{Harald2014}. Гарольд Хаас, которому принадлежит идея создания Li-Fi~\cite{Haas16}, предлагает использовать Li-Fi как комплиментарную сеть для облегчения нагрузки на РЧ спектр, так как значительная часть нагрузки на текущие Wi-Fi сети сможет быть перемещена на Li-Fi сети. 



  \mainmatter % это включает нумерацию глав и секций в документе ниже

  \chapter{Обзор системы VLC}

\section{Li-Fi и VLC}

\Abbrev{IM}{intensity modulation}
\Abbrev{DD}{direct detection}
\Abbrev{IEEE (Institute of Electrical and Electronics Engineers)}{институт электроники и инжинеров электроники}

Концепт VLC был представлен Гаральдом Хаасом на конференции TED Talk в 2011 году~\cite{Haas2011}. Было предложено использование модуляции интенсивности излучения \--- IM (intensity modulation), а для приёмника (фотодетектор) применять прямое детектирование \--- DD (direct detection). Работа такой системы описывается стандартом IEEE 802.15.7~\cite{IEEE2018}. VLC предлагается использовать как замену проводам и волокну для передачи данных~\cite{Haas16}, а именно для соединения формата <<от точки к точке>>, то есть предполагается, что каждый передатчик соединен только с одним приёмником.

Этот формат соединения и отличается VLC от Li-Fi: последний описывает сеть с двусторонней коммуникацией и наличием многих передатчиком и приёмников, предполагается соединение между одним источником и многими приёмниками \--- соединение формата <<от многих точек к точке>>. Такой тип соединения схож с уже существующим и широко распространенным протоколом Wi-Fi: он тоже позволяет обеспечить мобильность пользователей и незаметное переключение между передатчиками. По сути, стандарт Li-Fi включает в себя стандарт VLC, то проиллюстрировано на схеме~\ref{fig:vlcvslifi}.

\begin{figure}[!ht]
    \centering
    \includegraphics[width=.6\textwidth]{inc/img/vlcvslifi.png}
    \caption{Принципиальная схема Li-Fi и VLC~\cite{Haas16}}
    \label{fig:vlcvslifi}
\end{figure}

\section{Li-Fi передатчик}



Зачастую в системах передачи информации с помощью видимого света в качестве передатчика выбирается светодиодный светильник\footnote{LED luminaire}~\cite{LeMinh2008,Komine2006,Komine2004}. Он представляет из себя полноценное осветительное устройство, состоящее из LED источник излучения, балласта, корпуса и других компонентов. LED источник может состоять из одного или нескольких светодиодов, которые управляются с помощью управляющей микросхемы \--- контроллера, который контролирует ток, питающий светодиод и меняющий его яркость. Когда светодиодный светильник используется для коммуникации, контроллер модернизируется для передачи данных с помощью модуляции излучения. Примером простейшей модуляции является On-Off Keying, то есть <<нули>> и <<единицы>> передаются как два разных уровня интенсивности света.

Важнейшим преимуществом Li-Fi системы является возможность использования её для освещения вместо обычных светильников. В таком случае, функции передачи информации и освещения не должны мешать работе друг друга. Белый свет является превалирующим, так как при таком освещении цвет предметов выглядит естественным, как при солнечном освещении. Для освещений внутри и снаружи помещений всё чаще применяют LED светодиоды, так как они являются экономичными и надёжными. Тогда белый свет можно получить следующими способами:

\Abbrev{LD (laser diode)}{лазерный диод}

\begin{enumerate}
    \item Синий светодиод с фосфором \--- это источник света, состоящий из синего InGaN светодиода, покрытого жёлтым фосфором. Синий свет от источника поглощается фосфором и переизлучается на широком спектре от красного до зелёного, тем самым генерируя белый свет. Изменение цветовой температуры излучаемого света достигается за счёт изменения толщины фосфорного покрытия. 
    \item RGB светодиод \--- это три светодиода (красный, синий и зелёный), при смешении света от которых, получается белый свет. Этот тип светодиодов подходит для Li-Fi систем, чем фосфорные светодиоды, так как последние значительно ограничены временем релаксации фосфора, что может снизить скорость передачи данных (так как оказывается невозможно модулировать интенсивность такого светодиода с частотой выше нескольких МГц~\cite{Khalid2012}). Кроме того, при использовании RGB светодиодов возможно применение цветовой манипуляции \--- метода модуляции с разделением каналов по длинам волн, что позволяет повысить скорость передачи данных~\cite{Bian2019} (раздел \ref{CSK}).
    \item Лазерный диод (LD) \--- в 2011 году в лаборатории Sandia~\cite{Neumann2011} предложили использовать комбинацию четырёх цветных лазерных диодов (красный, синий, зелёный, жёлтый) для получения белого света. Авторы получили очень яркий собранный пучок света, который хорошо подходит для освещения мест, требующих высокую освещенность. Так как лазерное излучение может нанести вред человеку (и из-за коллимированности плохо освещает большую площадь), авторы использовали рассеиватели. Полученное излучение уже не имеет некоторых лазерных характеристик, однако лазерные диоды являются значительно более эффективными, чем RGB светодиоды. В~\cite{Hussein2015} авторы использовали три лазерных диода (рисунок~\ref{fig:ld}) для создания источника света для VLC системы. Они смогли добиться скорости передачи данных до 5 Гб/с. 
\end{enumerate}

\begin{figure}[!ht]
    \centering
    \includegraphics[width=.7\textwidth]{inc/img/ld.png}
    \caption{Принципиальная схема источника в VLC системе в~\cite{Hussein2015} с использованием лазерных диодов. В левой части схемы изображены три цветных лазера: красный, синий и зелёный соответственно; в правой части: разделители пучка и детекторы, устройства для совмещения лучей, зеркало, рассеиватель. На выходе получается свет белого цвета.}
    \label{fig:ld}
\end{figure}

\section{Li-Fi приёмник}

В качестве приёмников в Li-Fi системах чаще всего используются

\begin{enumerate}
    \item фотодетекторы \--- фотодиоды,
    \item датчики изображения \--- камеры.
\end{enumerate}

Фотодетектор \--- полупроводниковое устройство, которое генерирует ток при падении на него света. Современные коммерческие фотодетекторы могут детектировать с частотой до десятков МГц. 

\Abbrev{FPS (frames per second)}{кадры в секунду}

Кроме фотодетекторов, возможно применение камер, которые уже есть в большинстве техники (смартфоны, планшеты, ноутбуки), что может упростить интеграцию этих устройств в Li-Fi систему. Кроме того, возможно применение Li-Fi для передачи информации в рамках интернета вещей~\cite{Duquel2018}. Возможность использовать потребительские камеры появляется из-за того, что камера представляет из себя матрицу фотодиодов. Важным отличием является их количество, что делает быстрый сбор и обработку информации с них затруднительным. Это связано со снимаемой камерой частотой кадров в секунду. Несмотря на значительные ограничения по пропускной способности, есть способы повышения скорости приёма информации, например \--- использование эффекта Rolling shutter~\cite{TRAN2018}. Этот эффект заключается в следующем: так как количество пикселей в камере велико, считывание информации с них происходит построчно (или по столбцам). Если модулировать излучение с периодом, меньшим, чем время считывания одного столбца (строки), то в каждом столбце (строке) будет информация о передаче символа.

Тем не менее, использование фотодиодов остается более предпочтительным, так как с использованием их возможно достижение скорости передачи данных в несколько Гб/с~\cite{Cossu2012}.



  \chapter{Типы компонентов}

\section{Фотодиоды}

Выбор компонентов для определённого сценария использования определяется требованиями, выдвигаемыми к данной системе. Для фотодиода это может быть: интервал длин волн, которые он может детектировать, скорость отклика, фоточувствительность, сила темнового тока, тип корпуса. Это важно, так как существует значительное количество стандартных компонентов с различными стандартными корпусами, материалами полупроводника, характеристиками. Например, рассмотрим некоторые фотодиоды, представленные на сайте производителя Thorlabs (рисунок~\ref{fig:thorlabs}).

\begin{figure}[!h]
    \centering
    \includegraphics[width=.6\textwidth]{inc/img/thorlabs.png}
    \caption{Некоторые виды полупроводниковых PIN-фотодиодов, представленных на сайте производителя Thorlabs}
    \label{fig:thorlabs}
\end{figure}

\begin{enumerate}
    \item DSD2 \--- двух-зонный фотодиод из Si/InGaAs, кремний находится над InGaAs, за счёт этого фотодиод имеет очень широкий интервал детектируемых длин волн (400\--1100 нм для Si и 1000\--1800 нм для InGaAs), размеры фотодиода: 5.07 мм${}^2$ для Si, 1.77 мм${}^2$ для InGaAs, тип корпуса TO-5;
    \item FDG05 \--- фотодиод из Ge, интервал детектируемых длин волн 800\--1800 нм, размеры фотодиода: 19.6 мм${}^2$, выращен на керамической подложке, поэтому не имеет конкретного типа корпуса;
    \item FDS10X10 \--- большой кремниевый фотодиод с интервалом детектируемых длин волн 340\--1100 нм, с размерами фотодиода 100 мм${}^2$. Аналогично предыдущему выращен на керамической подложке;
    \item FGA01 \--- InGaAs фотодиод с высокой скоростью, типовым корпусом, интервал детектируемых длин волн 800\---1700 нм, размеры фотодиода 0.01 мм${}^2$, имеет стандартный корпус TO-46 и сферическую линзу;
    \item FGA01FC \--- тот же предыдущий фотодиод, но с корпусом, позволяющим подключение к оптоволокну;
    \item FGA21 \--- большой быстрый InGaAs фотодиод с интервалом детектируемых длин волн 800\--1700 нм, размерами фотодиода 3.1 мм${}^2$, тип корпуса TO-5;
    \item FDG03 \--- большой Ge фотодиод в стандартном корпусе TO-5, интервал детектируемых длин волн 800\--1800 нм, размер фотодиода 7.1 мм${}^2$.
\end{enumerate}

\section{Линзы}

Аналогично и для линз существуют различные параметры, которые влияют на характеристики и места применения конкретных линз: габаритные размеры, тип линзы (собирающая/рассеивающая), материал линзы (коэффициент преломления влияет на фокусное расстояние линзы), радиусы кривизны поверхностей (аналогично влияют на фокусное расстояние), толщина линзы по оси, наличие просветляющего покрытия и его материал (наличие покрытия повышает коэффициент пропускания линзы для конкретных длин волн, определяемых материалом покрытия).

Так как для улучшения приёмной части системы есть смысл собирать лучи, падающие на приёмник, была выбрана собирающая линза. Большинство её параметров не имеют значения, так как единственными ограничениями оказываются прозрачный на выбранной длине волны материал и фокусное расстояние, позволяющее разместить активную область фотодиода в фокусе (активная область находится внутри корпуса). Остальные параметры влияют на фокусное расстояние, однако от него будут зависеть габаритные размеры всего приёмного устройство в Li-Fi системе. Исходя из этих ограничений, для моделирования была выбрана выпукло-плоская собирающая линза из оптического стекла К8, прозрачного в ближнем ИК диапазоне. Более подробное описание параметров линзы приведено в разделе~\ref{ch:zemax}.
  \chapter{Методы модулирования излучения}

В VLC системах информация кодируется при помощи модуляции фазы, амплитуды и интенсивности. При рассмотрении различных схем модуляции необходимо не забывать об использовании системы в качестве освещения, так как некоторые параметры могут влиять на психо-физическое состояние человека. Некоторые такие параметры:

\begin{enumerate}
    \item затемнение \--- для разных сценариев освещения необходимы различные уровни освещенности~\cite{Zukauskas2002}. Если для освещения общественных пространств обычно достаточно света в интервале $30-100$ лк, то для освещения офисов и жилых помещений нужно освещение $300-1000$ лк. Современные светодиодные драйверы, управляющие питанием светодиодов, позволяют устанавливать любые необходимые уровни освещения в зависимости от сценария использования и требований по экономии энергии;
    \item смягчение мерцания \--- так как модуляция интенсивности подразумевает высокочастотное изменение интенсивности источника света, необходимо выбирать такой интервал частот, который не воспринимается человеческим глазом. Было показано (\cite{Berman1991}), что заметное мерцание освещения в течение продолжительного периода времени может привести к физиологическим последствиям у людей. Основной стандарт, описывающий системы VLC и Li-Fi \--- IEEE 802.15.7~\cite{IEEE2018} \--- рекомендует использовать частоту модулирования интенсивности не ниже 200 Гц.
\end{enumerate}

Рассмотрим четыре типа модуляции, которые применяются в VLC: 

\Abbrev{ИМ}{импульсная модуляция}
\Abbrev{CSM (color shift modulation)}{цветовая манипуляция}

\begin{enumerate}
    \item On-Off Keying (OOK);
    \item импульсная модуляция (ИМ);
    \item мультиплексирование с ортогональным частотным разделением каналов;
    \item цветовая манипуляция (CSM);
\end{enumerate}

\section{On-Off Keying}

\Abbrev{NRZ}{non-return-to-zero [on-off-keying]}

Самым простым методом модуляции излучения является On-Off Keying (OOK). Биты данных <<1>> и <<0>> здесь кодируются включенным и выключенным состоянием светодиода. На самом деле, не обязательно полностью выключать светодиод, достаточно лишь уменьшения интенсивности его излучения до некого порогового уровня. Такой тип модуляции часто применяется в волоконных коммуникациях. При использовании такой модуляции с синим фосфорным светодиодом, скорость передачи данных будет значительно ограниченна (из-за времени релаксации энергетических уровней фосфора \--- частота модуляции может быть не выше нескольких МГц~\cite{Grubor2007}). Если же использовать OOK, при котором светодиод не выключается, а его интенсивности уменьшается до порогово значения, как было описано выше, то возможно получение пропускной способности порядка десятков Мб/c~\cite{Park2007}. Если же использовать синий фильтр, чтобы убрать из сигнала свет жёлтого фосфора, то можно повысить пропускную способность до 40 Мб/c~\cite{Grubor2007}. Аналогично в~\cite{Minh2008,Vucic2009} было предложено комбинирование синего фильтра и аналогового выравнивания на приёмнике для достижения скорости передачи данных 100 и 125 Мб/с соответственно. Если использовать лавинный фотодиод (а не p-i-n фотодиод), то можно ещё больше повысить пропускную способность~\cite{Vucic2010}. Это связано с тем, что лавинный светодиод обладает более высокой чувствительностью и может регистрировать малые световые мощности. В таком случае пропускная способность системы возрастает до 230 Мб/с~\cite{Vucic2010}. 

Эта схема модуляции может быть использована и с RGB светодиодом, который имеет более быстрый отклик. В~\cite{Fujimoto2013} было продемонстрирована схема для передачи данных с RGB светодиодом, OOK схемой модуляции и p-i-n фотодиодом в качестве фотоприёмника. Достигнутая скорость передачи данных составила 477 Мб/с.

\section{Методы импульсной модуляции}

\Abbrev{PWM (pulse width modulation)}{модуляция длительности импульса}
\Abbrev{PPM (pulse position modulation)}{фазово-амплитудная модуляция}

Несмотря на то, что OOK имеет ряд преимуществ (простота и лёгкость реализации), оно имеет значительное ограничение \--- низкая скорость передачи данных. Поэтому были разработаны альтернативные методы модуляции, которые основаны на длительности (PWM) и положении импульса (PPM). 

\subsection{Модуляция длительности импульса}
\label{PWM}

Модуляция длительности импульса (PWM) является эффективным методом модуляции с помощью затемнения. Импульсы несут кодированный сигнал, а длительность импульсов определяет уровень освещенности. Из-за этого возможно изменять уровень освещенности без изменения интенсивности импульсов, так как сигнал кодируется длительностью импульса, во время которого светодиод работает на постоянной мощности. К минусам PWM относится достаточно низкая скорость передачи информации (4.8 Кб/с в~\cite{Sugiyama2007}).

\subsection{Фазово-амплитудная модуляция}

Фазово-амплитудная модуляция (PPM) основана на фазе импульса. В этой схеме модуляции длительность символа разделена на несколько интервалов одинаковой длительности, в одном из которых находится импульс. Тогда кодирование информации происходит с помощью положения импульса в конкретном интервале. Так как этот метод модуляции является достаточно простым, он был одним из первых, применённых в VLC системах \--- \cite{Georghiades1994,Shiu1999}. Существуют различные вариации этой схемы, например для передачи данных в условиях плохого соединения можно использовать PPM с адаптивной частотой передачи и повторением сигналов~\cite{Gfeller1996}.

\Abbrev{OPPM (overlapping pulse position modulation)}{перекрывающаяся фазово-амплитудная модуляция}

% START FROM HERE

Так как PPM предполагает передачу только одного импульса за временной интервал, это приводит к низкой скорости передачи данных и спектральной эффективности (спектральная эффективность \--- это скорость передачи данных, с которой возможно передавать информацию через конкретную полосу пропускания, характеризует эффективность использования частотного диапазона). Чтобы преодолеть эти проблемы были предложены альтернативные варианты PPM \--- перекрывающиеся фазово-амплитудные модуляции (OPPM), в которых, как следует из названия, возможно передавать несколько импульсов за временной интервал~\cite{Shiu1999}. Кроме того, возможно и перекрывание нескольких символов (смотри рисунок~\ref{fig:ppmvspwm}). Использование OPPM позволяет иметь более детальный контроль над уровнем освещения, решает проблему низкой спектральной эффективности и скорости передачи данных по сравнению с OOK и PPM~\cite{BoBai2010}.

\begin{figure}[!ht]
    \centering
    \includegraphics[width=.5\textwidth]{inc/img/ppmvspwm.png}
    \caption{Схематическая диаграмма, показывающая различия между модуляцией длительности импульса (PWM), фазово-амплитудной модуляцией (PPM), переменной фазово-амплитудной модуляцией (VPPM), перекрывающейся фазово-амплитудной модуляцией (OPPM) и многоимпульсной фазово-амплитудной модуляцией (MPPM). $S_n$ обозначает n-ный символ.~\cite{Pathak2015}}
    \label{fig:ppmvspwm}
\end{figure}

\Abbrev{MPPM (multipulse pulse position modulation)}{многоимпульсная фазово-амплитудная модуляция}
В~\cite{Sugiyama1989} была предложена другая схема, основанная на PPM, которая аналогично OPPM даёт возможность передавать несколько импульсов за интервал длительности символа. В отличии от OPPM, однако, здесь импульс не обязательно должен быть непрерывным (смотри рисунок~\ref{fig:ppmvspwm}). Это позволяет повысить спектральную эффективность по сравнению с OPPM~\cite{Shiu1999}.

\Abbrev{VPPM (variable pulse position modulation)}{переменная фазово-амплитудная модуляция}

В самом стандарте, описывающем системы VLC и Li-Fi, IEEE 802.15.7~\cite{IEEE2018} предлагается альтернативная схема модуляции \--- переменная фазово-амплитудная модуляция (VPPM), которая имеет ряд общих черт с PPM и PWM. Как и в первой, данные кодируются фазой импульса, в то время как возможно изменение длительности импульса при необходимости. В результате этого, VPPM имеет высокую надёжность и простоту, и позволяет детально варьировать уровень освещения, как и в PWM~\ref{PWM}.

\subsection{Мультиплексирование с орто­гональным частотным разделением каналов (OFDM)}

Преимущество мультиплексирования с орто­гональным частотным разделением каналов (OFDM), которое широко применяется в РЧ-коммуникациях, является отсутствие межсимвольной интерференции, которая может появляться в описанных выше методах модуляции из-за нелинейности частотного отклика каналов связи. Было предложено использовать OFDM и для систем передачи данных по видимому свету~\cite{Afgani2006}. Особенностью OFDM является то, что канал разделяется на несколько орто­гональных несущих, данные по которым передаются параллельно потоками,  модулируемыми по несущим. Проблема применения OFDM для беспроводных систем связи по видимому свету заключается в том, что OFDM генерирует комплексные биполярные сигналы, которые необходимо сконвертировать в действительные сигналы.

Другой проблемой OFDM является нелинейность отношения приложенного тока к интенсивности света LED~\cite{Burchardt2014}. Это выражается в отношении пиковой к средней мощности и было изучено в~\cite{Elgala2009,Elgala2009}. Там авторы предлагают в качестве решения использование светодиода в небольшом интервале мощности, на котором это отношение квазилинейное. 

Несмотря на эти недостатки, OFDM  имеет высокие перспективы, и было показано, что возможно достижение скорости передачи до нескольких Гб/с с использованием одного светодиода~\cite{Khalid2012,Tsonev2014}.

\subsection{Цветовая манипуляция (CSK)}
\label{CSK}

В стандарте IEEE 802.15.7~\cite{IEEE2018} описывается ещё один тип модуляции, который был разработан специально для использования в беспроводных системах связи по видимому свету \--- цветовая манипуляция (CSK). Суть этого метода заключается в отдельном модулировании трёх цветовых компонент RGB светодиода. 

\begin{figure}[!ht]
    \centering
    \includegraphics[width=.65\textwidth]{inc/img/cie1931.png}
    \caption{Цветовое пространство CIE 1931~\cite{IEEE2018}}
    \label{fig:cie1931}
\end{figure}

CSK модуляция основана на модели цветового пространства CIE 1931~\cite{CIE1931} (рисунок~\ref{fig:cie1931}, таблица~\ref{table:cie1931}), на которой всем видимым человеческим глазом цветам присвоена пара $(x,y)$ координат. Все это пространство разделено на семь полос, представленных в таблице~\ref{table:cie1931}

\begin{table}[!h]
    \centering
    \begin{tabular}{|c| c| c| c|} 
     \hline
     Полоса (нм) & Код & Центральная длина волны (нм) & $(x,y)$ \\ \hline
     $380-478$ & $(000)$ & $429$ & $(0.169, 0.007)$ \\ \hline
     $478-540$ & $(001)$ & $509$ & $(0.011, 0.733)$ \\ \hline
     $540-588$ & $(010)$ & $564$ & $(0.402, 0.597)$ \\ \hline
     $588-633$ & $(011)$ & $611$ & $(0.669, 0.331)$ \\ \hline
     $633-679$ & $(100)$ & $656$ & $(0.729, 0.271)$ \\ \hline
     $679-726$ & $(101)$ & $703$ & $(0.734, 0.265)$ \\ \hline
     $726-780$ & $(110)$ & $754$ & $(0.734, 0.265)$ \\ \hline
    \end{tabular}
    \caption{Семь полос, используемых в CSK, их коды, центральные длины волн и координаты на цветовом пространстве~\cite{IEEE2018}}
    \label{table:cie1931}
\end{table}

Суть алгоритма кодирования информации заключается в следующем~\cite{IEEE2018}:

\begin{enumerate}
    \item выбираются три точки в цветовом пространстве, которые соответствуют цветам RGB светодиода \--- получается треугольник;
    \item в зависимости от выбранной схемы кодирования (4-CSK, 8-CSK, 16-CSK \--- четыре, восемь и шестнадцать символов, соответственно) определяются координаты символов на плоскости. По сути это является задачей на оптимизацию, так как необходимо выбрать координаты так, чтобы расстояние между ними было наибольшим (это необходимо для минимизации интерференции между символами);
    \item по координатам символов вычисляется яркость компонент RGB светодиода:
    \begin{equation}
        \begin{gathered}
            x_s = P_i x_i + P_j x_j + P_k x_k \\
            y_s = P_i y_i + P_j y_j + P_k y_k \\
            P_i + P_j + P_k = 1
        \end{gathered}
    \end{equation}

    Здесь $(P_i, P_j, P_k)$ \--- яркости компонент светодиода, $(x_s, y_s)$ \--- цветовые координаты символа на цветовой плоскости (рисунок~\ref{fig:cie1931}), $(x_{ijk}, y_{ijk})$ \--- координаты центральных длин волн компонент светодиода на цветовой плоскости. 
\end{enumerate}

Для большего уменьшения межсимвольной интерференции возможно использовать светодиоды с четырьмя цветовыми компонентами (циан, синий, желтый, красный), так как тогда форма области на цветовой диаграмме будет четырёхугольником (а не треугольником, как в случае RGB светодиода), что позволит расположить символы дальше друг от друга~\cite{Singh2014}.


  \chapter{Моделирование оптической системы}
\label{ch:zemax}

\section{Базовая система}
\label{sec:basic_system}

В качестве источника излучения был выбран ИК лазерный диод FPL1055T с длиной волны излучения 1550 нм, мощностью излучения в импульсном режиме 300 мВт, поперечной расходимостью 28${}^\circ$ и боковой расходимостью 15${}^\circ$~\cite{LDThorlabs}. Выбранный фотодиод \--- FDGA05 с пиковой длиной волны 1550 нм (регистрируемый диапазон длин волн 800\--1700 нм), фоточувствительностью 0.95 А/Вт, площадью активной области 0.196 мм${}^2$ и материалом сенсора InGaAs~\cite{PDThorlabs}.

Для моделирования оптической системы использовался пакет Zemax OpticStudio 21.1.2. Рассмотрим задание параметров компонентов системы (рисунки~\ref{fig:no_lens_zemax_1}):

\begin{figure}[!h]
    \centering
    \includegraphics[width=\textwidth]{inc/img/no_lens_1.png}
    \caption{Задание параметров оптической системы в Zemax}
    \label{fig:no_lens_zemax_1}
\end{figure}

\begin{figure}[!h]
    \centering
    \includegraphics[width=.75\textwidth]{inc/img/layout.png}
    \caption{Вид системы сбоку; слева \--- площадка лазерного диода, справа \--- фотодиод в корпусе; расстояние между лазерным диодом и корпусом фотодиода 10~мм}
    \label{fig:layout}
\end{figure}

Первые 11 колонок (до колонки <<\# Layout Rays>>) универсальны для всех объектов в Zemax: тип объекта, комментарий, референтный объект, нахождение внутри другого объекта, X, Y и Z координаты, наклон вокруг осей X, Y и Z, материал объекта (где применимо). Далее идут специфические для объекта параметры:

Для лазерного диода: количество лучей для рендеринга и для расчётов (выбираются произвольно), мощность источника (выбрана в соответствии с~\cite{LDThorlabs}), номер длины волны (задаётся список используемых длин волн, задана только одна \--- 1550 нм, поэтому в ячейке стоит ноль \--- выбор любой длины волны из списка), цвет лучей на рендере, астигматизм (расстояние, на которое смещено распределение излучения в плоскости XZ), расходимость $\alpha_x$ для направления OX, супер-гауссов коэффициент $G_x$ в направлении OX, расходимость $\alpha_y$ и супер-гауссов коэффициент $G_y$ для направления OY. 

Супер-гауссов коэффициент показывает отличие профиля интенсивности данного пучка от профиля интенсивности Гауссова пучка \--- чем он больше, тем ближе профиль интенсивности излучения к прямоугольному профилю~\cite{Paschotta2008}. Для обоих направлений был выбран супер-гауссов коэффициент $G_{x,y} = 1$ (Гауссово распределение интенсивности). Следуя документации Zemax, можно рассчитать $\alpha_{x,y}$ следующим образом: 

\begin{equation*}
    \alpha_{x, y}=\frac{\theta_{\mathrm{fwhm}}}{\sqrt{2 \ln (2)}},
\end{equation*}

где $\theta_{\mathrm{fwhm}}$ \--- угол расходимости из~\cite{LDThorlabs}.

\begin{figure}[!h]
    \centering
    \includegraphics[width=.45\textwidth]{inc/img/pd_size.png}
    \caption{Виды сверху и спереди фотодиода FDGA05~\cite{PDThorlabs}}
    \label{fig:pdviews}
\end{figure}

Два следующих объекта формируют корпус фотодиода: круг с отверстием и трубка. Для круга с отверстием задаются радиусы отверстия и самого круга по большой и малой полуосям. Их значения соответствуют размерам реального лазерного диода~\cite{PDThorlabs}. Его положение задаётся относительно лазерного диода. Для трубки задаются радиусы начала и конца, длина. Радиусы равны и взяты из документации к фотодиоду, длина взята произвольной, положение задаётся относительно круга с отверстием. Этого достаточно для данного моделирования, так как стекло, закрывающее активную область от физического воздействия влияет пренебрежимо слабо на ход лучей и оптическую мощность на приёмнике, а всё, что находится за активной областью не влияет на модель вообще. Виды фотодиода приведены на рисунке~\ref{fig:pdviews}.

Для фотодиода заданы размеры апертуры (взяты из документации), количество угловых и радиальных зон оставлены по умолчанию и необходимы для расчётов в симуляции. Положение фотодиода задаётся относительно круга с отверстием и соответствует положению фотодиода в корпусе. Вид системы сбоку представлен на рисунке~\ref{fig:layout}.

Для анализа системы зададим 5$\cdot \text{10}^6$ лучей, проведём трассировку лучей и получим оптическую мощность, полученную на фотодиоде. Анализ будем производить для различных расстояний между лазерным диодом и фотоприёмником. Результаты всех симуляций приведены в приложении~\ref{ch:simulation_results} в таблице~\ref{tab:distance_simulation}. Построим график зависимости оптической мощности от расстояния (рисунок~\ref{fig:distance_plot}).

\begin{figure}[h]
    \centering
    \begin{minipage}[c]{.49\textwidth}
        \centering
        \includegraphics[width=\textwidth]{inc/img/distance.png}
    \end{minipage}
    \hfill
    \begin{minipage}[c]{.49\textwidth}
        \centering
        \includegraphics[width=\textwidth]{inc/img/distance2.png}
    \end{minipage}
    \caption{Зависимость оптической мощности на фотодиоде от расстояния между лазерным диодом и фотодиодом: результаты симуляции показаны синими точками, аппроксимирующая функция показана зелёной пунктирной линией}
    \label{fig:distance_plot}
\end{figure}

Эту зависимость можно аппроксимировать функцией $f(x) = \frac{1}{R^2}$, так как зависимость оптической мощности от расстояния подчиняется закону обратных квадратов. Лучшим образом для этого подходит функция $f(x) = \frac{0.615}{(x+1.639)^2} - 2.049\cdot10^{-4}$ (показана на графике пунктирной зелёной линией, результаты симуляции показаны синими точками).

% Аналогично исследуем зависимость оптической мощности от угла расходимости источника при расстоянии между источником и приёмником 100 мм. Выберем одинаковые углы расходимости по обеим осям, так как приёмник симметричен. Результаты симуляции приведены в приложении~\ref{ch:simulation_results} в таблице~\ref{tab:divergence_simulation}. График результатов приведён на рисунке~\ref{fig:divergence_plot}.

% \begin{figure}[h]
%     \centering
%     % This file was created by tikzplotlib v0.9.8.
\begin{tikzpicture}

\definecolor{color0}{rgb}{0.12156862745098,0.466666666666667,0.705882352941177}

\begin{axis}[
tick align=outside,
tick pos=left,
x grid style={white!69.0196078431373!black},
xlabel={Расходимость источника излучения, \(\displaystyle {}^\circ\)},
xmajorgrids,
xmin=-0.45, xmax=31.45,
xminorgrids,
xtick style={color=black},
y grid style={white!69.0196078431373!black},
ylabel={Оптическая мощность на фотодиоде, Вт},
ymajorgrids,
ymin=-0.00056135, ymax=0.01207435,
yminorgrids,
ytick style={color=black}
]
\addplot [semithick, color0, mark=*, mark size=3, mark options={solid}, only marks]
table {%
1 0.0115
3 0.00131
5 0.00047
8 0.000188
10 0.00012
13 7.4e-05
15 5.26e-05
18 3.38e-05
20 2.78e-05
23 2.3e-05
25 1.9e-05
28 1.47e-05
30 1.3e-05
};
\end{axis}

\end{tikzpicture}

%     \caption{Зависимость оптической мощности на фотодиоде от угла расходимости излучения лазерного диода: результаты симуляции показаны синими точками}
%     \label{fig:divergence_plot}
% \end{figure}

\section{Базовая система с собирающей линзой}

Далее добавим в систему собирающую сферическую линзу, конфигурация системы приведена на рисунке~\ref{fig:with_lens_zemax}. 

\begin{figure}[!h]
    \centering
    \includegraphics[width=\textwidth]{inc/img/with_lens.png}
    \caption{Задание параметров оптической системы с линзой в Zemax}
    \label{fig:with_lens_zemax}
\end{figure}

\begin{figure}[!h]
    \centering
    \includegraphics[width=.75\textwidth]{inc/img/layout_lens.png}
    \caption{Вид системы сбоку; слева \--- площадка лазерного диода, по середине \--- собирающая линза, справа \--- фотодиод в корпусе; расстояние между лазерным диодом и корпусом фотодиода 20~мм}
    \label{fig:layout_lens}
\end{figure}


Рассмотрим параметры линзы: зададим положение линзы относительно круга с отверстием, аналогично положению фотодиода. Материал линзы \--- оптическое стекло К8 с показателем преломления $n = 1.507$ на длине волны $\lambda = 1550$ нм. Линза плоско-выпуклая с радиусом кривизны поверхности $R = 8.5$ мм, толщиной линзы $d = 3.3$ мм. Очевидно, что фокусной расстояние этой линзы (по формуле \eqref{eq:focal_length}) $f = 16.76$ мм. 

\begin{equation}
    \frac{1}{f} = \frac{n-1}{R}
    \label{eq:focal_length}
\end{equation}

Тогда для лучшей фокусировки лучей необходимо поместить фотодиод в фокус линзы, то есть необходимо поместить линзу на расстоянии $15.86$ мм от корпуса фотодиода (фотодиод находится на расстоянии $0.9$ мм от поверхности корпуса) \--- рисунок~\ref{fig:with_lens_zemax}. Вид системы с боку представлен на рисунке~\ref{fig:layout_lens}. Аналогично двум предыдущим случаям, будем менять расположение линзы относительно фотодиода и наблюдать за изменением оптической мощности на приёмнике. Все результаты симуляции приведены в приложении~\ref{ch:simulation_results} в таблице~\ref{tab:lens_simulation}. График результатов приведен на рисунке~\ref{fig:lens_plot}.

\begin{figure}[h]
    \centering
    % This file was created by tikzplotlib v0.9.8.
\begin{tikzpicture}

\definecolor{color0}{rgb}{0.12156862745098,0.466666666666667,0.705882352941177}

\begin{axis}[
tick align=outside,
tick pos=left,
x grid style={white!69.0196078431373!black},
xlabel={Расстояние от линзы до фотодиода, мм},
xmajorgrids,
xmin=3.25, xmax=41.75,
xminorgrids,
xtick style={color=black},
y grid style={white!69.0196078431373!black},
ylabel={Оптическая мощность на фотодиоде, Вт},
ymajorgrids,
ymin=-9.571565e-05, ymax=0.00202646265,
yminorgrids,
ytick style={color=black}
]
\addplot [semithick, color0, mark=*, mark size=3, mark options={solid}, only marks]
table {%
5 7.47e-07
10 1.66e-06
15 0.00035
15.86 0.00193
20 0.000437
25 3.32e-05
30 7.06e-06
35 3.52e-06
40 2.3e-06
};
\end{axis}

\end{tikzpicture}

    \caption{Зависимость оптической мощности на фотодиоде от расстояния между собирающей линзой и фотодиодом: результаты симуляции показаны синими точками}
    \label{fig:lens_plot}
\end{figure}

По графику видно, что действительно наибольшая оптическая мощность на фотодиоде наблюдается при нахождении фотодиода в фокусе линзы. 

\section{Исследование диаграммы направленности приёмника}

Для более полного представления о характеристиках фотодиода составим нормированный график оптической мощности в зависимости от направления источника излучения. Для этого заменим лазерный диод на плоский прямоугольный источник (Source Rectangle в Zemax) \--- рисунок~\ref{fig:spinner_zemax}, вид системы с боку приведён на рисунке~\ref{fig:spinner_layout}. 

\begin{figure}[h]
    \centering
    \includegraphics[width=\textwidth]{inc/img/spinner_zemax.png}
    \caption{Задание параметров оптической системы с плоским прямоугольным источником в Zemax}
    \label{fig:spinner_zemax}
\end{figure}

\begin{figure}[!h]
    \centering
    \includegraphics[width=.75\textwidth]{inc/img/spinner_layout.png}
    \caption{Вид системы сбоку; слева \--- площадка плоского источника излучения, справа \--- фотодиод в корпусе; расстояние между лазерным диодом и корпусом фотодиода 20~мм}
    \label{fig:spinner_layout}
\end{figure}

Параметры задаются аналогично параметрам лазерного диода в разделе~\ref{sec:basic_system}. Зададим размеры источника 13x13 мм (квадрат со стороной, равной диаметру линзы). Для создания параллельного пучка излучения зададим супер-гауссовы параметры $G_{x,y}$: чем больше эти параметры, тем более узким выходит пучок. Были выбраны значения $G_{x,y} = 10^{20}$, в результате чего пучок лучей оказывается достаточно параллельным для выбранного расстояния~\cite{Zemax2021}:

\begin{equation}
    I(l, m) \approx I_{0} \exp{\left(-G_{x} l^{2}-G_{y} m^{2}\right)},  G_{x} = G_y \gg l,m \Rightarrow I(l, m) \approx I_0
\end{equation}

где $l$ и $m$ \--- направляющие косинусы вдоль $OX$ и $OY$ соответственно. Величина мощности не важна, так как будем нормировать графики по максимальному значению. Будем вращать источник вокруг $OX$ (в одну сторону, так как система полностью симметрична относительно плоскости $ZOX$; вращаем с шагом в $1^\circ$) и трассировать лучи. Результаты всех симуляций приведены в приложении~\ref{ch:simulation_results} в таблице~\ref{tab:spinner_with_lens}. Построим график зависимости оптической мощности от угла поворота источника (рисунок~\ref{fig:spinner_with_lens_plot}).

\begin{figure}[!h]
    \centering
    \includegraphics[width=.6\textwidth]{inc/img/spinner_with_lens_plot.png}
    \caption{График зависимости нормированной оптической мощности на приёмнике от угла поворота источника (система с линзой)}
    \label{fig:spinner_with_lens_plot}
\end{figure}

Аналогично проведём моделирование системы без линзы. Результаты всех симуляций приведены в приложении~\ref{ch:simulation_results} в таблице~\ref{tab:spinner_no_lens}, график зависимости оптической мощности от угла поворота источника представлен на рисунке~\ref{fig:spinner_no_lens_plot}.

\begin{figure}[!h]
    \centering
    \includegraphics[width=.6\textwidth]{inc/img/spinner_no_lens_plot.png}
    \caption{График зависимости нормированной оптической мощности на приёмнике от угла поворота источника (система без линзы)}
    \label{fig:spinner_no_lens_plot}
\end{figure}

По результатам изменения параметров видно, что самая высокая оптическая мощность достигается при наименьшем расстоянии между фотодиодом и лазерным диодом или при установке собирающей линзы на расстоянии от фотодиода, равном фокусному. Однако даже в наилучших случаях оптическая мощность значительно меньше, чем мощность излучения источника ($0.203$ Вт и $0.00193$ Вт соответственно, против $0.3$ Вт источника). Очевидно, что необходима дополнительная оптическая система для коллимации лучей источника и для фокусировки лучей у приёмника для уменьшения оптических потерь системы и для создания системы, в которой возможно передавать информацию.
  \chapter{Экспериментальные исследования системы}
\label{ch:experiment}

Как уже было сказано в предыдущем разделе~\ref{ch:zemax}, в качестве приёмника был выбран фотодиод FDGA05. Рассчитаем пропускную способность $f_{BW}$ этого фотодиода по данным производителя~\cite{PDThorlabs}:

\begin{equation}
    f_{BW} = \frac{1}{2\pi R_L C_j},
\end{equation}

где $R_L = 50$ Ом \--- сопротивление нагрузки, $C_j = 10$ пФ $= 10^{-11}$ Ф \--- ёмкость фотодиода. Тогда $f_{BW} = 3.1831 \cdot 10^8~\text{Гц} = 318.31~\text{МГц}$.

Экспериментально подтвердим это значение. Для этого соберём экспериментальную установку, состоящую из фотодиода FDGA05~\cite{PDThorlabs} и лазерного диода FPL1055T~\cite{LDThorlabs}. Оба компоненты распаяны на платы для подключения к векторному анализатору цепей Rohde & Schwarz ZVA 40 через SMA разъем (собранная установка аналогично установке представлена на рисунке~\ref{fig:experiment_setup_photo}: на ней показаны лазерный диод FPL1055T слева и фотодиод, использованный в работе~\cite{Kozyreva2019}).

\begin{figure}[!h]
    \centering
    \includegraphics[width=.9\textwidth]{inc/img/experiment_setup.jpg}
    \caption{Собранная экспериментальная установка с лазерным диодом FPL1055T и фотодиодом, использованным в работе~\cite{Kozyreva2019}}
    \label{fig:experiment_setup_photo}
\end{figure}

К первому порту векторного анализатора цепей подключен лазерный диод, ко второму \--- фотодиод. Для расчёта пропускной способности необходимо измерить ёмкость фотодиода экспериментально. Сделаем это при помощи измерения комплексного коэффициента отражения векторным анализатором (полученная диаграмма Смита приведена на рисунке~\ref{fig:smith_plot}).

\begin{figure}[!h]
    \centering
    \includegraphics[width=.9\textwidth]{inc/img/smith_plot.jpg}
    \caption{Диаграмма Смита для фотодиода}
    \label{fig:smith_plot}
\end{figure}

По измеренным данным не удаётся демодулировать сигнал из-за большого коэффициента отражения. Это связано с тем, что в сверх высоко частотный тракт ничего не проходит, так как нет согласования фотодиода по высоким частотам. % Для примера, на рисунке приведена диаграмма Смита для фотодиода, использованного в работе~\cite{Kozyreva2019}

  \backmatter %% Здесь заканчивается нумерованная часть документа и начинаются ссылки и

  \Conclusion

Результатом выпускной квалификационной работы являются модель оптической системы в программе Zemax OpticStudio. В ходе работы был проведен обзор существующей коммерчески доступной компонентной базы, на примере производителя Thorlabs, выбран фотодетектор, построена модель оптической системы с ним, проведено исследование зависимости оптической мощности на нём в зависимости от различных параметров системы, был проведён эксперимент с целью рассчитать ширину полосу пропускания фотодиода. 

В ходе работы было показано, что возможно создание Li-Fi сети с использованием стандартных компонентов, что позволит увеличить скорость внедрения и популяризации технологии, инфракрасной длины волны для восходящего сигнала, что является безопасным для человека, не мешает в освещении и позволяет увеличить мощность источника излучения для повышения качества передаваемого сигнала и скорости передачи информации. Была построена модель оптической системы, которая позволяет исследовать зависимость оптической мощности, улавливаемой фоточувствительной площадкой фотоприёмника в зависимости от различных параметров системы \--- углы расходимости источника, угол поворота источника, расстояние между источником и приёмником, мощность источника.

В дальнейшем работу можно продолжить: при использовании схемы согласования, возможно измерить ёмкость фотодиода и рассчитать его полосу пропускания, исследовать возможность и целесообразность использования светофильтров и просветляющего покрытия на оптике (фокусирующая линза, защитное стекло фотоприёмника) для улучшения качества связи или уменьшения необходимой мощности излучения источника. %% заключение

  %
% Список литературы при помощи BibTeX
%

% Книги без цитирования, которые будут отображаться в конце списка
% \nocite{kotlinInAction, cleanCode, gofPatterns, koldayev, gitProfessional, rxJava}

\bibliographystyle{ugost2008}
\bibliography{main}


  \appendix % Тут идут приложения

  \chapter{Результаты симуляции}
\label{ch:simulation_results}

\begin{table}
    \caption{Результаты симуляции оптической системы в Zemax, зависимость оптической мощности на фотодиоде от расстояния между лазерным диодом и фотодиодом}
    \label{tab:distance_simulation}
    \begin{tabularx}{\textwidth} {
        | l
        | >{\centering\arraybackslash}X
        | >{\centering\arraybackslash}X | }
        \hline
            № & Расстояние между лазерным диодом и фотодиодом, мм & Оптическая мощность на фотодиоде, Вт \\ \hline
            $1$ & $0.1$ & $2.03 \cdot 10^{-1}$ \\ \hline
            $2$ & $1$ & $8.82 \cdot 10^{-2}$ \\ \hline
            $3$ & $10$ & $3.33 \cdot 10^{-3}$ \\ \hline
            $4$ & $20$ & $8.97 \cdot 10^{-4}$ \\ \hline
            $5$ & $50$ & $1.49 \cdot 10^{-4}$ \\ \hline
            $6$ & $100$ & $3.70 \cdot 10^{-5}$ \\ \hline
            $7$ & $150$ & $1.73 \cdot 10^{-5}$ \\ \hline
            $8$ & $500$ & $1.80 \cdot 10^{-6}$ \\ \hline
            $9$ & $1000$ & $4.80 \cdot 10^{-7}$ \\ \hline
            $10$ & $2000$ & $6.00 \cdot 10^{-8}$ \\ \hline                    
    \end{tabularx}
\end{table}

% \begin{table}
%     \caption{Результаты симуляции оптической системы в Zemax, зависимость оптической мощности на фотодиоде от угла расходимости лазерного излучения лазерного диода на расстоянии 100 мм}
%     \label{tab:divergence_simulation}
%     \begin{tabularx}{\textwidth} {
%         | l
%         | >{\centering\arraybackslash}X
%         | >{\centering\arraybackslash}X | }
%         \hline
%             № & Угол расходимости излучения лазерного диода, ${}^\circ$ & Оптическая мощность на фотодиоде, Вт \\ \hline
%             $1$ & $1$ & $1.15 \cdot 10^{-2}$ \\ \hline
%             $2$ & $3$ & $1.31 \cdot 10^{-3}$ \\ \hline
%             $3$ & $5$ & $4.70 \cdot 10^{-4}$ \\ \hline
%             $4$ & $8$ & $1.88 \cdot 10^{-4}$ \\ \hline
%             $5$ & $10$ & $1.20 \cdot 10^{-4}$ \\ \hline
%             $6$ & $13$ & $7.40 \cdot 10^{-5}$ \\ \hline
%             $7$ & $15$ & $5.26 \cdot 10^{-5}$ \\ \hline
%             $8$ & $18$ & $3.38 \cdot 10^{-5}$ \\ \hline
%             $9$ & $20$ & $2.78 \cdot 10^{-5}$ \\ \hline
%             $10$ & $23$ & $2.30 \cdot 10^{-5}$ \\ \hline
%             $11$ & $25$ & $1.90 \cdot 10^{-5}$ \\ \hline
%             $12$ & $28$ & $1.47 \cdot 10^{-5}$ \\ \hline
%             $13$ & $30$ & $1.30 \cdot 10^{-5}$ \\ \hline                 
%     \end{tabularx}
% \end{table}

\begin{table}
    \caption{Результаты симуляции оптической системы в Zemax, зависимость оптической мощности на фотодиоде от расстояния между собирающей линзой и фотодиодом}
    \label{tab:lens_simulation}
    \begin{tabularx}{\textwidth} {
        | l
        | >{\centering\arraybackslash}X
        | >{\centering\arraybackslash}X | }
        \hline
            № & Расстояние между линзой и фотодиодом, мм & Оптическая мощность на фотодиоде, Вт \\ \hline
            $1$ & $5$ & $7.47 \cdot 10^{-7}$ \\ \hline
            $2$ & $10$ & $1.66 \cdot 10^{-6}$ \\ \hline
            $3$ & $15$ & $3.50 \cdot 10^{-4}$ \\ \hline
            $4$ & $15.86$ & $1.93 \cdot 10^{-3}$ \\ \hline
            $5$ & $20$ & $4.37 \cdot 10^{-4}$ \\ \hline
            $6$ & $25$ & $3.32 \cdot 10^{-5}$ \\ \hline
            $7$ & $30$ & $7.06 \cdot 10^{-6}$ \\ \hline
            $8$ & $35$ & $3.52 \cdot 10^{-6}$ \\ \hline
            $9$ & $40$ & $2.30 \cdot 10^{-6}$ \\ \hline           
    \end{tabularx}
\end{table}

\begin{table}
    \caption{Результаты симуляции оптической системы в Zemax, зависимость оптической мощности на фотодиоде от угла поворота источника (система с линзой)}
    \label{tab:spinner_with_lens}
    \begin{tabularx}{\textwidth} {
        | l
        | >{\centering\arraybackslash}X
        | >{\centering\arraybackslash}X | }
        \hline
            № & Угол поворота источника, ${}^\circ$ & Оптическая мощность на фотодиоде, Вт \\ \hline
            $1$ & $0$ & $1.82 \cdot 10^{-1}$ \\ \hline
            $2$ & $1$ & $8.47 \cdot 10^{-2}$ \\ \hline
            $3$ & $2$ & $3.42 \cdot 10^{-2}$ \\ \hline
            $4$ & $3$ & $2.74 \cdot 10^{-2}$ \\ \hline
            $5$ & $4$ & $2.41 \cdot 10^{-2}$ \\ \hline
            $6$ & $5$ & $1.50 \cdot 10^{-2}$ \\ \hline
            $7$ & $6$ & $3.42 \cdot 10^{-3}$ \\ \hline
            $8$ & $7$ & $0$ \\ \hline
    \end{tabularx}
\end{table}

\begin{table}
    \caption{Результаты симуляции оптической системы в Zemax, зависимость оптической мощности на фотодиоде от угла поворота источника (система без линзы)}
    \label{tab:spinner_no_lens}
    \begin{tabularx}{\textwidth} {
        | l
        | >{\centering\arraybackslash}X
        | >{\centering\arraybackslash}X | }
        \hline
            № & Угол поворота источника, ${}^\circ$ & Оптическая мощность на фотодиоде, Вт \\ \hline
            $1$ & $0$ & $1.13 \cdot 10^{-2}$ \\ \hline
            $2$ & $1$ & $1.13 \cdot 10^{-2}$ \\ \hline
            $3$ & $2$ & $1.13 \cdot 10^{-2}$ \\ \hline
            $4$ & $3$ & $1.13 \cdot 10^{-2}$ \\ \hline
            $5$ & $4$ & $1.13 \cdot 10^{-2}$ \\ \hline
            $6$ & $5$ & $1.13 \cdot 10^{-2}$ \\ \hline
            $7$ & $6$ & $1.13 \cdot 10^{-2}$ \\ \hline
            $8$ & $7$ & $1.13 \cdot 10^{-2}$ \\ \hline
            $9$ & $8$ & $1.12 \cdot 10^{-2}$ \\ \hline
            $10$ & $9$ & $1.12 \cdot 10^{-2}$ \\ \hline
            $11$ & $10$ & $1.12 \cdot 10^{-2}$ \\ \hline
            $12$ & $11$ & $1.11 \cdot 10^{-2}$ \\ \hline
            $13$ & $12$ & $1.11 \cdot 10^{-2}$ \\ \hline
            $14$ & $13$ & $1.10 \cdot 10^{-2}$ \\ \hline
            $15$ & $14$ & $1.10 \cdot 10^{-2}$ \\ \hline
            $16$ & $15$ & $1.09 \cdot 10^{-2}$ \\ \hline
            $17$ & $16$ & $1.09 \cdot 10^{-2}$ \\ \hline
            $18$ & $17$ & $1.08 \cdot 10^{-2}$ \\ \hline
            $19$ & $18$ & $6.62 \cdot 10^{-3}$ \\ \hline
            $20$ & $19$ & $0$ \\ \hline
    \end{tabularx}
\end{table}
\end{document}
