\Introduction

\Abbrev{РЧ}{радио частоты}
\Abbrev{VLC (visible light communication)}{связь по видимому свету}
\Abbrev{LiFi}{Light fidelity}
\Abbrev{ТД}{точка доступа}

В современном мире радио-частотный (РЧ) спектр зачастую оказывается заполнен, из-за чего становится все сложнее использовать беспроводную передачу данных в условиях возрастающих требований пропускной способности и количества передаваемого трафика. В следствие этого, исследуются возможности использования высокочастотного диапазона для передачи данных. Связь по видимому свету (VLC) \--- одно из наиболее многообещающих решений, которое позволяет достичь скорости передачи данных до 10 Гб/с~\cite{Carreira2020}. На основе технологии VLC была предложена беспроводная сеть доступа, известная так же как LiFi. По сравнению с VLC, LiFi является более уникальной и мультифункциональной системой, которая позволяет использовать множество координированных источников света внутри помещений как точки доступа (ТД). Каждая ТД позволяет создать соединение нескольким мобильным пользователям~\cite{Haas16}. Из-за того, что сигнал является направленным и не может пройти через непрозрачные объекты, LiFi сети являются более защищенными внутри помещений по сравнению с традиционными РЧ сетями~\cite{Cho2018}.