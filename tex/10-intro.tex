\Introduction

\Abbrev{РЧ}{радио частоты}
\Abbrev{VLC (visible light communication)}{связь по видимому свету}

Спустя тридцать лет после появления первых коммерческих мобильных коммуникационных систем, беспроводная связь эволюционировала в обыкновенное удобство как газ или электричество. Экспоненциальный рост в мобильном трафике в течение последних трёх десятилетий привет к масштабному разворачиванию беспроводных систем. Как следствие, ограниченный доступный РЧ-диапазон является объектом постоянного переиспользования и межканальной интерференции, что значительно ограничивает ёмкость сетей. Таким образом, было много различных предупреждений о грядущем <<кризисе РЧ спектра>>~\cite{Ofcom2013}, так как требования к передачи мобильных данных данных продолжают расти, в то время как спектральная эффективность сетей насыщается, даже несмотря на введение новых стандартов и значительный технологический прогресс в этой области. По оценкам, к 2022 году более $77$ эксабайтов ($10^{60}$ байт) трафика будет передаваться через мобильные устройства каждый месяц (около одного зеттабайта в год)~\cite{Cisco2019}. В середине прошлого десятилетия было предложено использование связи по видимому свету (VLC) в качестве потенциального решения для избежания <<кризиса РЧ спектра>>.

\Abbrev{ИК}{инфракрасный}
\Abbrev{LED (Light emitting diode)}{светодиод}
\Abbrev{AP (Access point)}{точка доступа}
\Abbrev{PLC (Power-line communication)}{связь по электросети}
\Abbrev{PoE (Power over Ethernet)}{питание через Ethernet}


В течение прошлого десятилетия значительные усилия были направлены на изучение альтернативных частей электро-магнитного спектра, которые потенциально смогут разгрузить б\'ольшую часть трафика из загруженного РЧ диапазона. Было продемонстрировано использование миллиметровых волн в коммуникации в 28 ГГц  диапазоне, а так же использование видимого и ИК света. Это особенно полезно, так как освещение \--- удобство, которое имеется практически в любой жилой среде и для которого существует готовая инфраструктура. Использование видимого света для высокоскоростных линий связи становится возможно из-за использования LED. В этом смысле концепт комбинирования функций освещения и коммуникации позволяет экономить энергию (и деньги) и сократить углеродный след. Во-первых, установка точек доступа (AP) является достаточно тривиальной задачей, так как можно переиспользовать уже существующую инфраструктуру с использованием готовых технологий, таких как связь по электросети (PLC) и питание через Ethernet (PoE). Во-вторых, так как освещение обычно работает в помещениях даже в течение светлого времени суток, дополнительное питание передатчиков будет незначительным. Помимо этого, видимый свет включает в себя сотни ТГц свободного канала, что на четыре порядка больше, чем полный РЧ спектр до 30 ГГц, включая миллиметровый спектр. Оптическое излучение, в общем, не интерферирует с другими радио волнами и не мешает работе чувствительного электрического оборудования. Таким образом, свет идеален для беспроводного покрытия в местах, чувствительных к электро-магнитному излучению (например, больницы, самолёты, топливно-химические и атомные электростанции и другие). Помимо этого, так как свет не может проникать через непрозрачные поверхности (стены), создается более высокий уровень безопасности соединения. Эта же особенность может быть использована для избежания интерференции между двумя соседними сетями.

\Abbrev{RGB (Red-green-blue)}{красный-синий-зелёный}
\Abbrev{MIMO (Multiple-input-mulitple-outputs)}{несколько входов и выходов}
\Abbrev{IM/DD (Intensity modulation and direct detection)}{модуляция интенсивности и прямое детектирования}
\Abbrev{OFDM (Orthogonal frequency division multiplexing)}{мультиплексирование
с ортогональным частотным разделением каналов}
\Abbrev{OWC (Optical wireless communication)}{оптическая беспроводная коммуникация}
\Abbrev{Li-Fi}{light fidelity}
\Abbrev{Wi-Fi}{wireless fidelity}

В течение последних десяти лет, было проведено значительное количество исследований об улучшении скорости передачи между двумя устройствами с использованием существующих светодиодов в лабораторных условиях. В 2012 году была достигнута скорость передачи данных выше 1 Гб/с с использованием белых фосфорных LED~\cite{Khalid2012}, и 3.4 Гб/с с помощью красно-сине-зелёного (RGB) LED~\cite{Cossu2012}. Также была продемонстрирована~\cite{Azhar2013} схожая гигабитная система с белым фосфорным LED в виде матрицы 4 на 4 в конфигурации несколько входов и выходов (MIMO). Теоретическая структура для достижимой ёмкости модуляции интенсивности и прямого детектирования (IM/DD) с использованием мультиплексирования с ортогональным частотным разделением каналов (OFDM) была показана в~\cite{Dimitrov2013}. Для успешной реализации системы мобильной связи необходима готовая сетевая система. Это и есть то, что называется Li-Fi \--- сетевое мобильное высокоскоростное VLC решение для беспроводной связи~\cite{Harald2014}. Гарольд Хаас, которому принадлежит идея создания Li-Fi~\cite{Haas16}, предлагает использовать Li-Fi как комплиментарную сеть для облегчения нагрузки на РЧ спектр, так как значительная часть нагрузки на текущие Wi-Fi сети сможет быть перемещена на Li-Fi сети. 

